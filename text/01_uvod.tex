\chapter*{Úvod}
\addcontentsline{toc}{chapter}{Úvod}  %pridanie nadpisu do obsahu

Predstavme si situáciu konferenčného hovoru. 
Jeden z účastníkov potrebuje niečo urobiť, no jeho aktivita by bola hlučná, čím by rušil ostatných účastníkov a teda si vypne mikrofón. 
Samozrejme, neskôr keď chce niečo povedať, nezapne mikrofón a ostatní ho nepočujú.
Ako by sme sa mohli popísanej situácií vyhnúť? 
Čo tak zapínať a vypínať mikrofón automaticky podla toho, či človek sediaci pred kamerou rozpráva alebo nerozpráva.
Jedným z riešení by mohla byť detekcia reči (Voice activity detection - VAD) pomocou mikrofónu (Acoustic Voice Actividy Detection - AVAD).
Toto riešenie však nemusí byť najpresnejšie, napríklad by nemuselo detegovať tichý hlas alebo by detegovalo nechcený šum a podobne.
Vhodným zlepšením by mohlo byť pridanie VAD pomocou obrazu webovej kamery (Visual Voice Actividy Detection - VVAD).
VVAD by mohlo priniesť viacero vylepšení, napríklad: odstránenie detekcie nechcených zvukov a šumu.

V práci sa zaoberáme možnosťami detekcie tváre v obraze, detekciou bodov na tvári a detekciou reči všeobecne. 
Rozoberáme existujúce riešenia danej problematiky a snažíme sa navrhnúť a implementovať vlastné riešenia.

Cieľom práce je analyzovať existujúce prístupy k riadeniu SW produktov pomocou HCI komponentov. 
Následne navrhneme vlastné HCI komponenty a pokúsime sa ich vhodným spôsobom ich implementovať.
Nakoniec sa budeme snažiť pilotne implementovať vytvorené komponenty do existujúcich SW produktov.

V prvej kapitole sa zaoberáme pojmom Human-Computer Interface a jeho históriou.
V druhej opisujeme existujúce riešenia detekcie reči.
V tretej je analyzovaný návrh riešenia, možnosti VVAD a AVAD a ich existujúce implementácie.
Detailný popis našej implementácie sa nachádza v kapitole 4.
Posledná kapitola obsahuje výsledky testovania nami implementovaného riešenia VAD.