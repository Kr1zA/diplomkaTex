\chapter*{Záver}
\addcontentsline{toc}{chapter}{Záver}  %pridanie nadpisu do obsahu
V práci sme sa zaoberali prvkami HCI, popísali sme ich históriu a aktuálne možnosti.
Zameriavame sa na detekciu reči a hľadali možnosti využitia tejto detekcia na automatické vypínanie a zapínanie mikrofónu pri videokonferenčnom hovore.
Vytvorili sme prehľad existujúcich riešení v nastolenej problematike.
Na základe naštudovanej literatúry sme navrhli a implementovali dynamickú C++ knižnicu VVAD, ktorú je možné použiť na detekciu tvárí v obraze kamery a na detekciu reči u osoby pred kamerou.
Nakoniec sme knižnicu otestovali a ukazuje sa ako reálne použiteľná. 
Do budúcna je možné:
\begin{itemize}
\item vylepšiť detekciu reči pomocou detekcie zo zvuku,
\item urobiť knižnicu viac modulárnou - rozdeliť ju na viac knižníc, kde jedna by detegovala tváre, ďalšia reč a podobne, 
\item nájsť rýchlejšiu a presnejšiu možnosť detekcie tvárí,
\item vytvoriť presnejší model detekcie bodov na tvári, \ldots
\end{itemize}